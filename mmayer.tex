%%%%%%%%%%%%%%%%%
% This is an example CV created using altacv.cls (v1.6.2, 28 Aug 2021) written by
% LianTze Lim (liantze@gmail.com), based on the
% Cv created by BusinessInsider at http://www.businessinsider.my/a-sample-resume-for-marissa-mayer-2016-7/?r=US&IR=T
%
%% It may be distributed and/or modified under the
%% conditions of the LaTeX Project Public License, either version 1.3
%% of this license or (at your option) any later version.
%% The latest version of this license is in
%%    http://www.latex-project.org/lppl.txt
%% and version 1.3 or later is part of all distributions of LaTeX
%% version 2003/12/01 or later.
%%%%%%%%%%%%%%%%

%% Use the "normalphoto" option if you want a normal photo instead of cropped to a circle
% \documentclass[10pt,a4paper,normalphoto]{altacv}

\documentclass[10pt,a4paper,ragged2e,withhyper]{altacv}

%% AltaCV uses the fontawesome5 package.
%% See http://texdoc.net/pkg/fontawesome5 for full list of symbols.

% Change the page layout if you need to
\geometry{left=1.25cm,right=1.25cm,top=1.5cm,bottom=1.5cm,columnsep=1.2cm}

% The paracol package lets you typeset columns of text in parallel
\usepackage{paracol}


% Change the font if you want to, depending on whether
% you're using pdflatex or xelatex/lualatex
\ifxetexorluatex
  % If using xelatex or lualatex:
  \setmainfont{Lato}
\else
  % If using pdflatex:
  \usepackage[default]{lato}
\fi

% Change the colours if you want to
\definecolor{BlueStyle}{HTML}{3973AC}
\definecolor{SlateGrey}{HTML}{2E2E2E}
\definecolor{LightGrey}{HTML}{666666}
% \colorlet{name}{black}
\colorlet{tagline}{BlueStyle}
\colorlet{heading}{BlueStyle}
\colorlet{headingrule}{BlueStyle}
% \colorlet{subheading}{PastelRed}
\colorlet{accent}{BlueStyle}
\colorlet{emphasis}{SlateGrey}
\colorlet{body}{LightGrey}

% Change some fonts, if necessary
% \renewcommand{\namefont}{\Huge\rmfamily\bfseries}
% \renewcommand{\personalinfofont}{\footnotesize}
% \renewcommand{\cvsectionfont}{\LARGE\rmfamily\bfseries}
% \renewcommand{\cvsubsectionfont}{\large\bfseries}

% Change the bullets for itemize and rating marker
% for \cvskill if you want to
\renewcommand{\itemmarker}{{\small\textbullet}}
\renewcommand{\ratingmarker}{\faCircle}

%% Use (and optionally edit if necessary) this .cfg if you
%% want to use an author-year reference style like APA(6)
%% for your publication list
\input{pubs-authoryear}

%% Use (and optionally edit if necessary) this .cfg if you
%% want an originally numerical reference style like IEEE
%% for your publication list
% \input{pubs-num.cfg}

%% sample.bib contains your publications
\addbibresource{sample.bib}

\begin{document}
\name{Michael Balzer}
\tagline{Solutions Engineer \& Proud Geek}
% Cropped to square from https://en.wikipedia.org/wiki/Marissa_Mayer#/media/File:Marissa_Mayer_May_2014_(cropped).jpg, CC-BY 2.0
%% You can add multiple photos on the left or right
\photoR{2.5cm}{1624468977526}
% \photoL{2cm}{Yacht_High,Suitcase_High}
\personalinfo{%
  % Not all of these are required!
  % You can add your own with \printinfo{symbol}{detail}
  \email{balzermw@gmail.com}
%   \phone{000-00-0000}
  \location{Sacramento, CA}
  \linkedin{michael-balzer}
  \github{balzermw} 
%   \orcid{0000-0000-0000-0000} % Obviously making this up too.
  %% You can add your own arbitrary detail with
  %% \printinfo{symbol}{detail}[optional hyperlink prefix]
  % \printinfo{\faPaw}{Hey ho!}
  %% Or you can declare your own field with
  %% \NewInfoFiled{fieldname}{symbol}[optional hyperlink prefix] and use it:
  % \NewInfoField{gitlab}{\faGitlab}[https://gitlab.com/]
  % \gitlab{your_id}
	%%
  %% For services and platforms like Mastodon where there isn't a
  %% straightforward relation between the user ID/nickname and the hyperlink,
  %% you can use \printinfo directly e.g.
  % \printinfo{\faMastodon}{@username@instace}[https://instance.url/@username]
  %% But if you absolutely want to create new dedicated info fields for
  %% such platforms, then use \NewInfoField* with a star:
  % \NewInfoField*{mastodon}{\faMastodon}
  %% then you can use \mastodon, with TWO arguments where the 2nd argument is
  %% the full hyperlink.
  % \mastodon{@username@instance}{https://instance.url/@username}
}

\makecvheader

%% Depending on your tastes, you may want to make fonts of itemize environments slightly smaller
\AtBeginEnvironment{itemize}{\small}

%% Set the left/right column width ratio to 6:4.
\columnratio{0.6}

% Start a 2-column paracol. Both the left and right columns will automatically
% break across pages if things get too long.
\begin{paracol}{2}

\cvsection{Experience}

\cvevent{Enterprise Solutions Engineer}{VMware}{May 2021 - Present}{Palo Alto, CA}
\begin{itemize}
\item Run discovery workshops to identify business outcomes, IT strategy and challenges from C-Level to administrators across IT and other Lines of Business.
\item Generate comprehensive future state architecture presentations to the C-suite
\item Technically representing the solution to the customer’s Enterprise architecture team as necessary.
\end{itemize}

\divider

\cvevent{Solutions Engineer}{Ayehu}{June 2020 -- May 2021}{San Francisco, CA}
\begin{itemize}
\item  Function as an automation evangelist by providing custom technical solutions to match unique business needs.
\item Present Ayehu’s value via live webinars and Q/A.
\item Prepare SOW documents alongside sales team and guide prospects through entire sales funnel- Discovery to Closed.
\end{itemize}

\divider

\cvevent{Sales Engineer}{KeepTruckin}{November 2019 --  June 2020}{San Francisco, CA}
\begin{itemize}
\item  Acting as a technical product expert for the sales team in the form of technical demonstrations, solution architecture and proof of concepts for a SAAS product.
\item Work alongside PM and engineering team to provide feedback on IOT device and AI camera product improvements.
\item Develop and present training materials to equip sales team with differentiated product information.
\end{itemize}
\divider

\cvevent{Customer Support Engineer}{Samsara}{December 2017 -- June 2020}{San Francisco, CA}

\begin{itemize}
\item Supported Samsara's enterprise customer base through Hypergrowth-- growing from 500 to 15,000 customers.
\item  Kickstarted Samsara's customer knowledge base by drafting and editing technical articles.
\item Utilize GraphQL, Python, REST API & basic shell commands to health check critical accounts \& troubleshoot Samsara hardware.
\end{itemize}

% \divider

% \cvevent{Product Engineer}{Google}{23 June 1999 -- 2001}{Palo Alto, CA}

% \begin{itemize}
% \item Joined the company as employe \#20 and female employee \#1
% \item Developed targeted advertisement in order to use user's search queries and show them related ads
% \end{itemize}




%% Switch to the right column. This will now automatically move to the second
%% page if the content is too long.
\switchcolumn

\cvsection{Life Philosophy}
\begin{quote}
``Learn by doing''
\end{quote}

\cvsection{Most Proud of}

\cvachievement{\faTrophy}{Courage I had}{to move into Solution Engineering from a support role.}

\divider

\cvachievement{\faCircle}{Tenacity}{To continue the lifelong journey of learning throughout my career.}

\divider

\cvachievement{\faChartLine}{Samsara's Growth}{Being able to participate and experience the excitement of hyper-growth.}

\divider

\cvachievement{\faUsers}{Inspiring SEs}{Through SE organizations like PSC, mentoring and building a more diverse world of presale professionals.}

\cvsection{Strengths}

\cvtag{Goal Oriented}
\cvtag{Tenacious}\\
\cvtag{Independent}

\divider\smallskip

\cvtag{IT Technology}
\cvtag{SAAS Solution Selling}
\cvtag{Executive Alignment}

\cvsection{Education}


\cvevent{B.S.\ in Geographic Information Systems}{California State University Sacramento}{Sept 2014 -- June 2016}{}


\end{paracol}

\end{document}
